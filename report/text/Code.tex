\section{Vue d'ensemble du projet}

\subsection{Features}

\begin{itemize}
	\item Configuration de l'intégralité des serveurs avec pairs.cfg.
	\item Connexion de plusieurs clients.
	\item Utiliser du MultiThreading côté client afin de gérer les messages asychrones,
	\item Faire le relai des messages en fonction de leur type par les serveurs esclaves.
	\item Utilisation du design pattern Template.
\end{itemize}

NOTE: Les connexions directes au serveur central ne sont pas autorisées. Cela aurait été possible à l'aide du multithreading en créant un thread chatAmuCentral responsable des connexions des clients, mais dans la mesure où cela revient à la même chose que de connecter un autre serveur esclave au serveur maître, l'idée n'a pas été retenue.

\subsection{Bugs}

\begin{itemize}
\item Aucune demande d'authenfication. Plusieurs personnes peuvent donc s'identifier sous le même login.
\item Les utilisateurs déconnectés ne sont pas retirés de la liste des clients, ce qui peut entrainer un dédoublement des messages.
\end{itemize}

\subsection{Améliorations possibles}

\begin{itemize}
	\item utiliser le pattern Chains of responsibility, ce qui permettrait d'améliorer la souplesse de l'implémentation du protocole, et la réduction de la taille du code de chatAmuCentral,
	\item une gestion de mots de passe chiffrés, afin d'améliorer la sécurité,
\end{itemize}
