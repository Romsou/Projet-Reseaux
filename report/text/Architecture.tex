\section{L'architecture du code}
Il y a 4 grand types de pacquets dans le projet:
\begin{itemize}
    \item Le paquet Client, qui contient un client classique pouvant se connecter aux serveurs.
    \item Le paquet Config, donnant la configuration de base(adresse ip et port) du serveur fédéré maître.
    \item Le paquet Serveur, qui contient le code des serveurs.
    \item Le paquet Tools, qui contient tout les outils utilisé par les serveurs et les clients pour communiquer.
\end{itemize}

Le fichier ServerLauncher.java lance seulement un serveur chatAmuCentral, alors que le fichier FederationLauncher.java lance le serveur maître et les serveurs esclaves conformément au contenu de config.cfg.

\subsection{Serveur}
Ce paquet contient tout les serveurs, que ce soit les serveurs de chat(ChatAmu), ou le serveur fédéré(Federation).

Tout les fichiers de ce paquet étendent la classe abstraite TemplateServer, qui correspond à la base d'un serveur, sauf la gestion des clés de connections, qui sont précisé par les sous classes, ChatAmuCentral.java et MasterServer.java.

\subsection{Tools}
Ce paquet contient tout les outils crées pour assurer la communication entre les serveurs et les clients.

Communication contient la classe IOCommunicator.java, qui est chargée de la lecture et de l'écriture dans les buffers des serveurs.

ConfigParser permet de récupérer les données du fichiers de config, afin de pouvoir conntecter les serveurs au serveur maître.

Extended permet la gestion des erreurs du projet, via les codes d'erreur de la classe ErrorCodes.java, qui permet de lier les erreurs des autres classes à un code définit par le projet, et permettant de trouver la source de l'erreur.

Protocol contient la classe ProtocolHandler.java, qui sert à reconnaître les messages envoyés au serveur, et à différencier une connection d'un message au serveur.

UserManagment représente la base de clients connectés au serveur. Register sert  leur connection, et ClientQueueManager permet la gestion des clients, ansi que la transmission des messages qu'ils envoient.

\subsection{La conception}
Il est important de remarquer que l'architecture du projet respecte les bonnes pratiques de la programmation orientée objet, c'est à dire la délégation, au lieu de l'extension.
